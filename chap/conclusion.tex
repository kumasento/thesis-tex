% Copyright (c) 2014,2016 Casper Ti. Vector
% Public domain.

\specialchap{结论}

本研究采用了Xilinx公司推出的Zynq-7000全可编程SoC平台,结合深度学习框架Caffe,设计并实现了SoCaffe这一针对嵌入式平台的深度学习框架。最终实现的SoCaffe系统充分利用了Zynq平台的特点,可编程逻辑上实现的GEMM加速算法占用了大部分的硬件资源,基本取得了极限的性能;处理系统部分编译改善了Caffe工具,保持了与Caffe一致的使用方式。因此SoCaffe是一个针对Zynq SoC平台的实际可用的深度学习框架。

本研究的主要工作如下:
\begin{enumerate}
	\item 使用Vivado HLS工具实现了GEMM计算在FPGA上的设计与实现,针对Zynq-7000的系统资源特点给出了相应的优化方案与数学分析模型,找到了能实现的矩阵块大小的上界;同时使用了半精度浮点数优化了对计算资源的使用。最终获得最高12.31GFLOPS和5.42倍加速比。
	\item 系统地学习和掌握了Xilinx SDSoC开发环境的使用方式,通过文档与探索自主掌握了该工具的高级使用方法,并针对SoCaffe的特点定制了一套开发流程。
	\item 使用ARM GNU工具链使用交叉编译的方式编译了全部Caffe需要使用的第三方库,使用SDSoC构建了SoCaffe的系统镜像文件包和动态链接库。
	\item 测试了SoCaffe的计算性能,可用性,精度等等特性,找到了SoCaffe的最佳使用条件:即针对大量使用卷积神经网络,网络规模较大的时候,SoCaffe可以取得不错的加速比。
\end{enumerate}

此外,SoCaffe也存在一些缺点和亟待提高的地方:首先,因为Zynq SoC平台的内存有限,SoCaffe无法支持特别大的神经网络结构;其次,SoCaffe的GEMM固定块算法依然有一定的比重是冗余计算,如果输入矩阵形状比较小,则会造成较大比例的性能损失;最后,SoCaffe目前只支持GEMM的优化,其他不用到GEMM操作的网络层不能得到相应的性能提升。这些工作都是本研究下一步要进行优化的工作。



% vim:ts=4:sw=4

% Copyright (c) 2014,2016 Casper Ti. Vector
% Public domain.

\chapter{致谢}

本科四年的学习生活即将结束,随着这篇论文的完成,这段时光也终于画上了圆满的句号。在此,衷心对帮助过我的老师,同学和朋友们表示感谢。

首先,我要感谢本科四年的导师梁云老师。梁老师为人随和耐心、治学严谨,从我大一下冒昧申请参加科研实习以来,是他慢慢引导我走上了计算机科学研究的道路,教授我研究的方法和理念,指导推荐我参加各种科研实习与实践活动。在我大二的时候,梁老师就让我参与到真正的科研工作中,与国外的研究团队一起挑战高难度的课题。梁老师还热心帮我推荐到国外的科研机构参加研究工作,学习领先的技术和理念。此外,对于我的各种人生中的选择,梁老师都能认真地聆听和理解,并热心地给我建议。总之,这四年来梁老师给我的帮助,不仅让我能在本文的写作中受益匪浅,也让我对今后的科研生活有了明晰的目标,做好了充分的准备。

然后,我要感谢北京大学高能效计算与应用中心(CECA)的老师和师兄师姐们。CECA对我来说是一个充满归属感的地方:魏学超、谢小龙、李秀红、王硕等师兄在我的科研和本论文的写作中都给予了我充分的帮助和指导,而且忍耐我频繁的打扰和犯的错误;王韬老师、罗国杰老师、孙广宇老师和许辰人老师都在我四年的实习过程中给过我帮助和启发,四位老师的课程也都让我获益匪浅,帮我打下了坚实的基础并克服诸多难关。

我也要感谢四年来帮助过我、和我一起奋战过的同学和朋友们,是他们教会了我如何协作、互相帮助、共同进步。

最后我要感谢一直关爱我的家人致以诚挚的感谢。感谢父母二十多年来对我的培养和支持,对我的理解和义无反顾的帮助,让我能勇敢地追求心中的梦想。我也要感谢陪伴我四年的女朋友,你与我分享了这四年来的欢笑、泪水、焦虑和憧憬,给我单调乏味的人生带来绚丽的色彩,带给我冷静与细致谨慎的性格。是他们的爱和支持让我能成为今天的自己,我也会用我的余生来回报他们的恩情。

% vim:ts=4:sw=4
